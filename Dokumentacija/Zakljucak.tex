\chapter{Zaključak i budući rad}

Primarni cilj ovog projekta bio je razviti aplikaciju koja omogućuje građanima da ugovore šetnje pasa u skloništima za pse i tako pomognu u socijalizaciji pasa i povećaju vjerojatnost udomljavanja. Iz projektnog zadatka bilo je potrebno izlučiti funkcionalne i nefunkcionalne zahtjeve, konceptualno osmisliti i dokumentirati aplikaciju a zatim provesti implementaciju osmišljenog.
		
Provedba projekta bila je podijeljena u dva ciklusa. U prvom ciklusu oformili smo projektni tim, uspostavili kanale komunikacije i započeli s radom. Bilo nam je važno upoznati se i pokušati stvoriti dobru radnu atmosferu, usprkos tome što je komunikacija uživo bila otežana. Za brzu komunikaciju smo odabrali WhatsApp, a organizaciju projekta radili smo na stranici Trello koja podržava kanban metodu organizacije. Vrlo intenzivnim fokusom na razradu zahtjeva odmah na početku rada, stvorili smo čvrstu bazu za daljnji rad i osigurali da je svim članovima tima jasna slika aplikacije koju izrađujemo. 
		
U drugom ciklusu veći naglasak bio je na samoj implementaciji aplikacije. S obzirom na to da su svim članovima tima tehnologije s kojima smo radili bile dosad nepoznate, u drugom ciklusu se pokazala posvećenost i inicijativa članova. Svi članovi su samostalno istraživali tehnologije i tražili rješenja za probleme koji su se pojavljivali. Također je do izražaja došla važnost komunikacije i surađivanja. Nakon nekoliko nesporazuma i udvostručavanja odrađenog posla, stavili smo veći naglasak na poboljšanje komunikacije i učestalo izvještavanje o problemima koje treba riješiti za napredak. Počeli smo koristiti Discord koji nam je omogućio da više timova radi istovremeno na različitim komponentama i međusobno se brzo posavjetuje.
		
Radom na projektu svi smo stekli nova znanja i zato bismo da sada krećemo s projektom neke probleme vjerojatno riješili na brži i kvalitetniji način. U budućoj nadogradnji aplikacije vjerojatno bi započeli s refaktoriranjem pojedinih dijelova koda koji su nekonzistentni ili smo iskustvom rada na drugim komponentama uvidjeli da se mogu bolje riješiti. Osim toga, kao moguće proširenje funkcionalnosti aplikacije dodali bismo mogućnost grupnih šetnja s drugim ljudima, što mislimo da bi bilo zanimljivo i za prijatelje i za upoznavanje novih ljudi.

Svim članovima tima je ovaj projekt pružio nova znanja i vještine s kojima se dosad nismo susreli na FER-u. Osim tehničkih znanja iz korištenih tehnologija, stekli smo vrijedno iskustvo timskog rada i općenito samostalnog rada na projektu.
		
		\eject